% ******************************************************************************
% ****************************** Custom Margin *********************************

% Add `custommargin' in the document class options to use this section
% Set {innerside margin / outerside margin / topmargin / bottom margin}  and
% other page dimensions
\ifsetCustomMargin
\RequirePackage[left=37mm,right=30mm,top=35mm,bottom=30mm]{geometry}
  \setFancyHdr % To apply fancy header after geometry package is loaded
\fi

% Add spaces between paragraphs
%\setlength{\parskip}{0.5em}
% Ragged bottom avoids extra whitespaces between paragraphs
\raggedbottom
% To remove the excess top spacing for enumeration, list and description
%\usepackage{enumitem}
%\setlist[enumerate,itemize,description]{topsep=0em}

% *****************************************************************************
% ******************* Fonts (like different typewriter fonts etc.)*************

% Add `customfont' in the document class option to use this section

\ifsetCustomFont
  % Set your custom font here and use `customfont' in options. Leave empty to
  % load computer modern font (default LaTeX font).
  %\RequirePackage{helvet}

  % For use with XeLaTeX
  %  \setmainfont[
  %    Path              = ./libertine/opentype/,
  %    Extension         = .otf,
  %    UprightFont = LinLibertine_R,
  %    BoldFont = LinLibertine_RZ, % Linux Libertine O Regular Semibold
  %    ItalicFont = LinLibertine_RI,
  %    BoldItalicFont = LinLibertine_RZI, % Linux Libertine O Regular Semibold Italic
  %  ]
  %  {libertine}
  %  % load font from system font
  %  \newfontfamily\libertinesystemfont{Linux Libertine O}
\fi

% *****************************************************************************
% **************************** Custom Packages ********************************

\usepackage[utf8]{inputenc}
\usepackage{helvet}
\renewcommand{\familydefault}{\sfdefault}
\usepackage{geometry}
\usepackage{lipsum}

% ************************* Mathematics **************************

\usepackage{array}
\usepackage{amsmath}
\usepackage{amssymb}
\usepackage{bm}
\usepackage{amsfonts}
\usepackage{amsthm}
\usepackage{mathrsfs} 
\usepackage{amsopn}
\usepackage{mathtools}
\usepackage[linesnumbered, ruled,vlined,resetcount,algochapter]{algorithm2e}
\usepackage[nameinlink]{cleveref}

\newtheorem{theorem}{Theorem}[chapter]
\newtheorem{proposition}[theorem]{Proposition}
\newtheorem{lemma}[theorem]{Lemma}
\newtheorem{corollary}[theorem]{Corollary}
\newtheorem{remark}[theorem]{Remark}
\newtheorem{claim}[theorem]{Claim}
\newtheorem{example}[theorem]{Example}
\newtheorem{assumption}[theorem]{Assumption}
\newtheorem{definition}[theorem]{Definition}
\newtheorem{observation}[theorem]{Observation}

\Crefname{chapter}{Chap.}{Chaps.}
\Crefname{section}{Sect.}{Sects.}
\Crefname{proposition}{Prop.}{Props.}
\Crefname{theorem}{Thm.}{Thms.}
\Crefname{definition}{Defn.}{Defns.}
\Crefname{corollary}{Cor.}{Cors.}
\Crefname{figure}{Fig.}{Figs.}
\Crefname{observation}{Obs.}{Obss.}
\Crefname{assumption}{Asm.}{Asms.}
% ************************* Algorithms and Pseudocode **************************

%\usepackage{algpseudocode}


% ********************Captions and Hyperreferencing / URL **********************

% Captions: This makes captions of figures use a boldfaced small font.
%\RequirePackage[small,bf]{caption}
\RequirePackage[labelsep=space,tableposition=top]{caption}
\usepackage{subcaption}
%\renewcommand{\figurename}{Fig.} %to support older versions of captions.sty


% *************************** Graphics and figures *****************************

\usepackage{tikz}
\usepackage{pgfplots}
\usetikzlibrary{arrows.meta,positioning}
\usepackage{tikz-qtree,tikz-qtree-compat}
\usepackage{graphicx}
\usepackage{xcolor}
%\usepackage{rotating}
%\usepackage{wrapfig}

% Uncomment the following two lines to force Latex to place the figure.
% Use [H] when including graphics. Note 'H' instead of 'h'
%\usepackage{float}
%\restylefloat{figure}

% Subcaption package is also available in the sty folder you can use that by
% uncommenting the following line
% This is for people stuck with older versions of texlive
%\usepackage{sty/caption/subcaption}

% ********************************** Tables ************************************
\usepackage{booktabs} % For professional looking tables
\usepackage{multirow}
\usepackage{multicol}
\setlength{\columnseprule}{0pt}
\setlength\columnsep{10pt}
%\usepackage{longtable}
%\usepackage{tabularx}
\usepackage{verbatim}
\usepackage{multirow}
\usepackage{adjustbox}
\usepackage{longtable}
\usepackage{pdflscape}
\usepackage{url}        	% simple URL typesetting
\usepackage{subcaption}
\usepackage[inline]{enumitem}
\usepackage{diagbox}
\usepackage{cancel}



% *********************************** SI Units *********************************
\usepackage{siunitx} % use this package module for SI units


% ******************************* Line Spacing *********************************


\usepackage[absolute,overlay]{textpos}
% Choose linespacing as appropriate. Default is one-half line spacing as per the
% University guidelines

% \doublespacing
% \onehalfspacing
% \singlespacing


% ************************ Formatting / Footnote *******************************

% Don't break enumeration (etc.) across pages in an ugly manner (default 10000)
%\clubpenalty=500
%\widowpenalty=500

%\usepackage[perpage]{footmisc} %Range of footnote options

\usepackage{comment}
% *****************************************************************************
% *************************** Bibliography  and References ********************

%\usepackage{cleveref} %Referencing without need to explicitly state fig /table

% Add `custombib' in the document class option to use this section
\ifuseCustomBib
   \RequirePackage[square, sort, numbers, authoryear]{natbib} % CustomBib

% If you would like to use biblatex for your reference management, as opposed to the default `natbibpackage` pass the option `custombib` in the document class. Comment out the previous line to make sure you don't load the natbib package. Uncomment the following lines and specify the location of references.bib file

%\RequirePackage[backend=biber, style=numeric-comp, citestyle=numeric, sorting=nty, natbib=true]{biblatex}
%\addbibresource{References/references} %Location of references.bib only for biblatex, Do not omit the .bib extension from the filename.

\fi

% changes the default name `Bibliography` -> `References'
\renewcommand{\bibname}{References}



% *********** To change the name of Table of Contents / LOF and LOT ************

%\renewcommand{\contentsname}{My Table of Contents}
%\renewcommand{\listfigurename}{My List of Figures}
%\renewcommand{\listtablename}{My List of Tables}


% ********************** TOC depth and numbering depth *************************

\setcounter{secnumdepth}{2}
\setcounter{tocdepth}{2}


% ******************************* Nomenclature *********************************

% To change the name of the Nomenclature section, uncomment the following line

%\renewcommand{\nomname}{Symbols}


% ********************************* Appendix ***********************************

% The default value of both \appendixtocname and \appendixpagename is `Appendices'. These names can all be changed via:

%\renewcommand{\appendixtocname}{List of appendices}
%\renewcommand{\appendixname}{Appndx}

% *********************** Configure Draft Mode **********************************

% Uncomment to disable figures in `draft'
%\setkeys{Gin}{draft=true}  % set draft to false to enable figures in `draft'

% These options are active only during the draft mode
% Default text is "Draft"
%\SetDraftText{DRAFT}

% Default Watermark location is top. Location (top/bottom)
%\SetDraftWMPosition{bottom}

% Draft Version - default is v1.0
%\SetDraftVersion{v1.1}

% Draft Text grayscale value (should be between 0-black and 1-white)
% Default value is 0.75
%\SetDraftGrayScale{0.8}


% ******************************** Todo Notes **********************************
%% Uncomment the following lines to have todonotes.

%\ifsetDraft
%	\usepackage[colorinlistoftodos]{todonotes}
%	\newcommand{\mynote}[1]{\todo[author=kks32,size=\small,inline,color=green!40]{#1}}
%\else
%	\newcommand{\mynote}[1]{}
%	\newcommand{\listoftodos}{}
%\fi

% Example todo: \mynote{Hey! I have a note}

% ******************************** Highlighting Changes **********************************
%% Uncomment the following lines to be able to highlight text/modifications.
%\ifsetDraft
%  \usepackage{color, soul}
%  \newcommand{\hlc}[2][yellow]{{\sethlcolor{#1} \hl{#2}}}
%  \newcommand{\hlfix}[2]{\texthl{#1}\todo{#2}}
%\else
%  \newcommand{\hlc}[2]{}
%  \newcommand{\hlfix}[2]{}
%\fi

% Example highlight 1: \hlc{Text to be highlighted}
% Example highlight 2: \hlc[green]{Text to be highlighted in green colour}
% Example highlight 3: \hlfix{Original Text}{Fixed Text}

% *****************************************************************************
% ******************* Better enumeration my MB*************
\usepackage{enumitem}




% ******************************************************************************
% ************************* User Defined Commands ******************************
% ******************************************************************************

% abbreviations for words
\newcommand{\st}{s.t. }
\newcommand{\ie}{i.e., }
\newcommand{\eg}{e.g.,~}
\newcommand{\etc}{etc.}

% abbreivations for symbols
\newcommand{\bB}{\mathbb B}
\newcommand{\bR}{\mathbb R}
\newcommand{\bZ}{\mathbb Z}
\newcommand{\bQ}{\mathbb Q}
\newcommand{\bN}{\mathbb N}
\newcommand{\bP}{\mathbb{P}}

\newcommand{\cA}{\mathcal A}
\newcommand{\cB}{\mathcal B}
\newcommand{\cO}{\mathcal O}
\newcommand{\cC}{\mathcal C}
\newcommand{\cD}{\mathcal D}
\newcommand{\cE}{{\mathcal E}}
\newcommand{\cF}{\mathcal F}
\newcommand{\cL}{\mathcal L}
\newcommand{\cG}{\mathcal G}
\newcommand{\cK}{\mathcal K}
\newcommand{\cM}{\mathcal M}
\newcommand{\cI}{\mathcal I}
\newcommand{\cN}{\mathcal N}
\newcommand{\cX}{\mathcal X}
\newcommand{\cS}{\mathcal S}
\newcommand{\cU}{\mathcal U}
\newcommand{\cQ}{\mathcal Q}
\newcommand{\cV}{\mathcal V}
\newcommand{\cR}{\mathcal R}
\newcommand{\cY}{\mathcal Y}
\newcommand{\cZ}{\mathcal Z}
\newcommand{\cP}{\mathcal P}
\newcommand{\cEI}{{\mathcal {EI}}}
\renewcommand{\cI}{{\mathcal I}}
\newcommand{\deq}{\coloneqq}
\DeclareMathOperator{\suc}{s.t.}
\DeclareMathOperator{\proj}{proj}

\newcommand{\lin}[2]{{\Xi^{#1}_{#2}}}
\newcommand{\relx}[1]{\tilde{#1}}

\newcommand{\mst}{\textup{ s.t. }}
\renewcommand{\rm}[1]{\mathrm{#1}}


% math operators

\DeclareMathOperator{\dom}{dom}
\DeclareMathOperator{\rang}{range}
\DeclareMathOperator{\inter}{int}
\DeclareMathOperator{\relinter}{relint}
\DeclareMathOperator{\conv}{conv}
\DeclareMathOperator{\cone}{cone}
\DeclareMathOperator{\conve}{convenv}
\DeclareMathOperator{\epi}{epi}
\DeclareMathOperator{\hyp}{hypo}
\DeclareMathOperator{\gr}{gr}
\DeclareMathOperator{\gra}{gr}
\DeclareMathOperator{\cl}{cl}
\DeclareMathOperator{\bd}{bd}
\DeclareMathOperator{\dw}{DW}
\DeclareMathOperator{\argmin}{argmin}
\DeclareMathOperator{\argmax}{argmax}
\DeclareMathOperator*{\e}{e}
\DeclareMathOperator*{\vt}{v}
\DeclareMathOperator*{\ct}{ct}
\DeclareMathOperator{\sign}{sign}
\DeclareMathOperator{\nsign}{nsign}
\DeclareMathOperator{\prev}{prev}
\DeclareMathOperator{\dst}{dist}
\DeclarePairedDelimiter\ceil{\lceil}{\rceil}
\DeclarePairedDelimiter\floor{\lfloor}{\rfloor}
\newcommand{\norm}[1]{{\lVert#1\rVert}}

\newcommand{\vc}[1]{\bm{#1}}
\newcommand{\vx}{{\vc{x}}}
\newcommand{\va}{{\vc{\alpha}}}
\newcommand{\vy}{{\vc{y}}}
\newcommand{\vz}{{\vc{0}}}
\newcommand{\ve}{{\vc{z}}}
\newcommand{\vu}{{\vc{1}}}
\newcommand{\vv}{{\vc{v}}}
\newcommand{\vs}{{\vc{s}}}


\SetKwComment{Comment}{$\triangleright$\ }{}
\SetCommentSty{}
\SetKw{Continue}{continue}
\SetKw{Break}{break}



\newcommand{\rc}{{\text{rc}}}
\newcommand{\dfproblem}{discrete facility SCP}
\newcommand{\ddproblem}{discrete demand SCP}
\newcommand{\ncalgo}{\text{nodeCover}}
\newcommand{\malgo}{\text{mutual}}
\newcommand{\bO}[1]{{\mathcal{O}(#1)}}
\newcommand{\dash}{:}
\newcommand{\tabledefline}[2]{\multicolumn{1}{l}{\rlap{#1\ \dash\ #2}}\\}


\newcommand{\bsocpcomp}{BSOCP-BC\xspace}
\newcommand{\dwbc}{DW-BC\xspace}
\newcommand{\dwpwl}{DW-PWL\xspace}
\newcommand{\dwhybrid}{DW-Hybrid\xspace}
\newcommand{\dwhybrids}{DW-Hybrid*\xspace}
\DeclareMathOperator{\abs}{abs}



\newcommand{\couenne}{$\texttt{Couenne}$\xspace}
\newcommand{\baron}{$\texttt{BARON}$\xspace}
\newcommand{\antigone}{$\texttt{ANTIGONE}$\xspace}
\newcommand{\miso}{$\texttt{MISO}$\xspace}
\newcommand{\minlplib}{MINLPLib\xspace}
\newcommand{\scip}{SCIP\xspace}
\newcommand{\disable}{\texttt{disable}\xspace}
\newcommand{\ic}{\texttt{ic}\xspace}
\newcommand{\oc}{\texttt{oc}\xspace}
\newcommand{\oic}{\texttt{oic}\xspace}
\newcommand{\cSt}{\cS_{\mathrm{st}}}
\newcommand{\bcSt}{\overline{\cS}_{\mathrm{st}}}
\newcommand{\cSl}{\cS_{\mathrm{lift}}}

\newcommand{\maxcut}{\textsc{max cut}\xspace}
\newcommand{\pbm}{\textsc{pseudo Boolean maximization}\xspace}
\newcommand{\bdopt}{\textsc{Bayesian D-optimal design}\xspace}
\newcommand{\dopt}{\textsc{D-optimal design}\xspace}


\DeclareMathOperator{\relint}{relint}
\DeclareMathOperator{\relbd}{relbd}
\DeclareMathOperator{\rank}{rank}
\DeclareMathOperator{\var}{var}
\DeclareMathOperator{\trace}{trace}
\DeclareMathOperator{\ldet}{log\,det}
\DeclareMathOperator{\ext}{ext}
\newcommand{\cha}[1]{\mathsf{supp}(#1)}
\newcommand{\p}[1]{\ensuremath{\mathsf{PM}_{#1}} }
\newcommand{\ep}[1]{\ensuremath{\mathsf{EPM}_{#1}} }
\newcommand{\ee}[1]{\ensuremath{\mathsf{EE}_{#1} }}
\renewcommand{\t}[1]{{#1}^\top}
\newcommand{\sF}{{\mathsf F}}
\newcommand{\bsF}{\bar{\mathsf F}}

\newcommand{\bS}{{\mathbb S}}