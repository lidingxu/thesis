%%%%%%%%%%%%%%%%%%%%%%%%%%%%%%%%%%%%%%%%%%%%%%%%%%%%%%%%%%%%%%%%%%%%%%%%%%%%%%%%%%%%%%%%%%%%%%%%%%%%%%%%%%%%%%%%%%%%%%%%%%%%%%%%%%%%%%%%%%%%%%%%%%%%%%%%%%%%%%%%%%%%%%%
%%%%%%%%%%%%%%%%%%%%%%%%%%%%%%%%%%%%%%%%%%%%%%%%%%%%%%%%%%%%%%%%%%%%%%%%%%%%%%%%%%%%%%%%%%%%%%%%%%%%%%%%%%%%%%%%%%%%%%%%%%%%%%%%%%%%%%%%%%%%%%%%%%%%%%%%%%%%%%%%%%%%%%%
%%% Modèle pour la 4ème de couverture des thèses préparées à l'Institut Polytechnique de Paris, basé sur le modèle produit par Nikolas STOTT / Template for back cover of thesis made at Institut Polytechnique de Paris, based on the template made by Nikolas STOTT
%%% Mis à jour par Aurélien ARNOUX (École polytechnique)/ Updated by Aurélien ARNOUX (École polytechnique)
%%% Les instructions concernant chaque donnée à remplir sont données en bloc de commentaire / Rules to fill this file are given in comment blocks
%%% ATTENTION Ces informations doivent tenir sur une seule page une fois compilées / WARNING These informations must contain in no more than one page once compiled
%%%%%%%%%%%%%%%%%%%%%%%%%%%%%%%%%%%%%%%%%%%%%%%%%%%%%%%%%%%%%%%%%%%%%%%%%%%%%%%%%%%%%%%%%%%%%%%%%%%%%%%%%%%%%%%%%%%%%%%%%%%%%%%%%%%%%%%%%%%%%%%%%%%%%%%%%%%%%%%%%%%%%%%
%%% Version du 28 avril 2020 : utilisation de .png au lieu de .jpg pour les logos
%%%%%%%%%%%%%%%%%%%%%%%%%%%%%%%%%%%%%%%%%%%%%%%%%%%%%%%%%%%%%%%%%%%%%%%%%%%%%%%%%%%%%%%%%%%%%%%%%%%%%%%%%%%%%%%%%%%%%%%%%%%%%%%%%%%%%%%%%%%%%%%%%%%%%%%%%%%%%%%%%%%%%%%

\label{form_final}
%%%%%%%%%%%%%%%%%%%%%%%%%%%%%%%%%%%%%%%%%%%%%%%%%%%%%%%%%%%%%%%%%%%%%%%%%%%%%%%%%%%%%%%%%%%%%%%%%%%%%%%%%%%%%%%%%%%%%%%%%%%%%%%%%%%%%%%%%%%%%%%%%%%%%%%%%%%%%%%%%%%%%%%
%%%%%%%%%%%%%%%%%%%%%%%%%%%%%%%%%%%%%%%%%%%%%%%%%%%%%%%%%%%%%%%%%%%%%%%%%%%%%%%%%%%%%%%%%%%%%%%%%%%%%%%%%%%%%%%%%%%%%%%%%%%%%%%%%%%%%%%%%%%%%%%%%%%%%%%%%%%%%%%%%%%%%%%
%%% Formulaire / Form
%%% Remplacer les paramètres des \newcommand par les informations demandées / Replace \newcommand parameters by asked informations
%%%%%%%%%%%%%%%%%%%%%%%%%%%%%%%%%%%%%%%%%%%%%%%%%%%%%%%%%%%%%%%%%%%%%%%%%%%%%%%%%%%%%%%%%%%%%%%%%%%%%%%%%%%%%%%%%%%%%%%%%%%%%%%%%%%%%%%%%%%%%%%%%%%%%%%%%%%%%%%%%%%%%%%
%%%%%%%%%%%%%%%%%%%%%%%%%%%%%%%%%%%%%%%%%%%%%%%%%%%%%%%%%%%%%%%%%%%%%%%%%%%%%%%%%%%%%%%%%%%%%%%%%%%%%%%%%%%%%%%%%%%%%%%%%%%%%%%%%%%%%%%%%%%%%%%%%%%%%%%%%%%%%%%%%%%%%%%

\newcommand{\logoEd}{ed}																		%% Logo de l'école doctorale. Indiquer le sigle (EDIPP, EDMH) / Doctoral school logo. Indicate the acronym : EDMH, EDIPP
\newcommand{\PhDTitleFR}{Méthodes de relaxation pour la programmation non linéaire en nombres entiers mixtes}													%% Titre de la thèse en français / Thesis title in french
\newcommand{\keywordsFR}{Programmation non linéaire en nombres entiers mixtes, pptimisation globale, relaxation}														%% Mots clés en français, séprarés par des , / Keywords in french, separated by ,
\newcommand{\abstractFR}{\lipsum[1-3]}															%% Résumé en français / abstract in french

\newcommand{\PhDTitleEN}{Relaxation methods for mixed integer nonlinear programming}													%% Titre de la thèse en anglais / Thesis title in english
\newcommand{\keywordsEN}{Mixed-integer nonlinear programming, global optimization, relaxation}														%% Mots clés en anglais, séprarés par des , / Keywords in english, separated by ,
\newcommand{\abstractEN}{\lipsum[1-3]}															%% Résumé en anglais / abstract in english

\label{layout_final}

%%%%%%%%%%%%%%%%%%%%%%%%%%%%%%%%%%%%%%%%%%%%%%%%%%%%%%%%%%%%%%%%%%%%%%%%%%%%%%%%%%%%%%%%%%%%%%%%%%%%%%%%%%%%%%%%%%%%%%%%%%%%%%%%%%%%%%%%%%%%%%%%%%%%%%%%%%%%%%%%%%%%%%%
%%%%%%%%%%%%%%%%%%%%%%%%%%%%%%%%%%%%%%%%%%%%%%%%%%%%%%%%%%%%%%%%%%%%%%%%%%%%%%%%%%%%%%%%%%%%%%%%%%%%%%%%%%%%%%%%%%%%%%%%%%%%%%%%%%%%%%%%%%%%%%%%%%%%%%%%%%%%%%%%%%%%%%%
%%% Mise en page / Page layout      
%%% NE RIEN MODIFIER / DO NOT MODIFY
%%%%%%%%%%%%%%%%%%%%%%%%%%%%%%%%%%%%%%%%%%%%%%%%%%%%%%%%%%%%%%%%%%%%%%%%%%%%%%%%%%%%%%%%%%%%%%%%%%%%%%%%%%%%%%%%%%%%%%%%%%%%%%%%%%%%%%%%%%%%%%%%%%%%%%%%%%%%%%%%%%%%%%%
%%%%%%%%%%%%%%%%%%%%%%%%%%%%%%%%%%%%%%%%%%%%%%%%%%%%%%%%%%%%%%%%%%%%%%%%%%%%%%%%%%%%%%%%%%%%%%%%%%%%%%%%%%%%%%%%%%%%%%%%%%%%%%%%%%%%%%%%%%%%%%%%%%%%%%%%%%%%%%%%%%%%%%%
{
\thispagestyle{empty}
{\fontsize{9.2pt}{11.04pt}\selectfont
%%% Logo de l'école doctorale. Le nom du fichier correspond au sigle de l'ED / Doctoral school logo. Filename correspond to doctoral school acronym
%%% Les noms valides sont / Valid names are : EDMH, (EDIPP)
\tekstblokkulur{white}
\begin{textblock*}{61mm}(16mm,3mm)
	\noindent\includegraphics[height=24mm]{media/ed/EDIPP.png}
\end{textblock*}



%%%Titre de la thèse en français / Thesis title in french
\begin{center}
\fcolorbox{black}{white}{\parbox{0.95\textwidth}{
{\bf Titre:} Méthodes de relaxation pour la programmation non linéaire en nombres entiers mixtes
\medskip

%%%Mots clés en français, séprarés par des ; / Keywords in french, separated by ;
{\bf Mots clés:} PNLNE; optimisation globale ; relaxation
\vspace{-2mm}

%%% Résumé en français / abstract in french
\begin{multicols}{2}
{\bf Résumé:} 
Cette thèse se concentre sur la programmation non linéaire à variables mixtes (MINLP), une classe de problèmes d'optimisation mathématique, et les algorithmes associés pour les résoudre. L'algorithme central utilisé dans de nombreux solveurs d'optimisation globale pour les problèmes MINLP est l'algorithme  de séparation et évaluation. La clé du succès de l'algorithme  de séparation et évaluation réside dans l'utilisation de relaxations des problèmes d'optimisation, qui sont essentielles pour obtenir des bornes duales efficaces.
Cependant, la construction de relaxations efficaces dépend des structures spécifiques des problèmes d'optimisation. Dans la première partie de cette thèse, nous présentons un aperçu complet des outils de relaxation structurelle adaptés aux problèmes MINLP structurés liés à
différents domaines d’applications. Ces outils englobent des relaxations à partir de formulations étendues, des relaxations par sous-modularité, des relaxations utilisant une approximation linéaire par morceaux et des renforcements de relaxation via des coupes d'intersection. Nous développons de nouveaux résultats théoriques avancés basés sur ces outils. Dans la deuxième partie, nous utilisons ces techniques de relaxation pour aborder divers problèmes d'optimisation. Nous explorons les plans coupants pour la programmation signoïdale. Nous proposons des coupes d'intersection pour améliorer les relaxations  linéaire des problèmes d'optimisation sous-modulaire.
Nous étudions les relaxations de Dantzig-Wolfe pour un problème de programmation linéaire à variables mixtes dans le routage de réseaux sans fil et un problème MINLP dans le binpacking sous-modulaire. Enfin, nous étudions la technique de relaxation big-M appliquée aux fonctions linéaires par morceaux dans le problème de couverture continue sur un réseau. 
Les travaux réalisés durant cette thèse de doctorat contribuent à l’avancement des approches de la programmation non linéaire en nombre entiers et des méthodes d’optimisation connexes.
En effet la combinaison des études exhaustives réalisées sur diverses techniques de relaxation et leurs applications à différents contextes d’optimisation offrent des perspectives précieuses tant pour la compréhension théorique des problèmes que pour la mise en œuvre empirique des résultats.
\end{multicols}
}}
\end{center}

\vspace*{0mm}

%%%Titre de la thèse en anglais / Thesis title in english
\begin{center}
\fcolorbox{black}{white}{\parbox{0.95\textwidth}{
{\bf Title:} Relaxation methods for mixed-integer
nonlinear programming
\medskip

%%%Mots clés en anglais, séprarés par des ; / Keywords in english, separated by ;
{\bf Keywords:}  MINLP; global optimization; relaxation %%3 à 6 mots clés%%
\vspace{-2mm}
\begin{multicols}{2}
	
%%% Résumé en anglais / abstract in english
{\bf Abstract:} 
This thesis focuses on mixed-integer nonlinear programming (MINLP), a class of mathematical optimization problems, and the associated algorithms to solve them. The core algorithm utilized in many global optimization solvers for MINLP problems is the branch-and-bound algorithm. Key to the success of the branch-and-bound approach is the use of relaxations of optimization problems, which are vital in obtaining efficient and tight dual bounds.
However, constructing effective relaxations depends on the specific structures of optimization problems. In the first part of this thesis, we present a comprehensive overview of structural relaxation tools tailored for structured MINLP problems across different disciplines.  These tools encompass relaxations from extended formulations, relaxations via submodularity, relaxations using piece-wise linear approximation, and  relaxation tightening via intersection cuts. Then, we develop novel advanced theoretical results based on these tools. In the second part, we  employ these relaxation techniques to address various optimization problems. We explore cutting planes for signomial programming. Then, we propose intersection cuts for enhancing linear programming relaxations of submodular optimization problems.
Next, we investigate the Dantzig-Wolfe  relaxations for a mixed-integer linear programming problem in wireless network routing and a MINLP problem in submodular binpacking. Finally, we study the big-M relaxation technique as applied to piece-wise linear functions in the continuous covering problem on a network.  
By combining these comprehensive studies on various relaxation techniques and their applications in different optimization contexts, this thesis contributes to the advancement of MINLP and related optimization methods, offering valuable insights for both theoretical understanding and computational implementation.
\end{multicols}
}}
\end{center}

\tekstblokkulur{white}
\begin{textblock*}{161mm}(10mm,270mm)
{\bf\noindent Institut Polytechnique de Paris	         }

\noindent
\noindent 91120 Palaiseau, France 
\end{textblock*}

\tekstblokkulur{white}
\begin{textblock*}{20mm}(175mm,265mm)
\includegraphics[width=20mm]{media/IPPARIS-petit}
\end{textblock*}

}
}
